%This documents the Ballad VM. If we end up releasing it bare-bones, wrap this
%document in appropriate silly or challenge related stuff later.


\section{Ballad} Ballad is a programming language expressed in music. It is an
`interpreted', Turing complete language which does not follow the von Neumann
architecture.  This is the case insofar as code memory is separate from static,
non-executable memory as well as the actual, active memory. The Ballad VM has a
series of registers as well as a stack, in which active data is processed. 

The code memory cannot be modified during the execution of a program, nor can
the static memory section. These are not placed on a heap or stack, as in a
more conventional program, but are maintained separately from the active
memory. The static memory can be of arbitrary size, but the code memory is
restricted to a maximum of 256 bytes. The implementation of the Ballad VM
should reflect this.

The stack is similarly limited in size to 256 bytes. It can be pushed and
popped from, but it can also be addressed and written to directly in the style
of a stack of fixed length backed by an array. A stack pointer internal to the
VM should track the location of the top of the stack; the bottom is fixed to
the zero position and the stack grows upwards. The stack is addressed by Ballad
instructions ranging from 0x00 (the bottom) to 0xFF (the largest value).

All data is, by default, an unsigned eight bit integer. Modifications to this
may be made by the program (for example, implementing two's complement or the
like), but the Ballad VM should treat all numbers as such for the purpose of
calculating overflow.

There are two sets of registers which Ballad uses. These are the math registers
and the swap registers. One set is designed for and used by all arithmetic
operations, the other set is designed for moving data between locations and as
a general purpose register set. There are eight registers of each type and are
addressed by a byte of value 0x00 to 0x07. 

\subsection{Decoding} Ballad programs consist of static data sections and code
sections which are expressed as music in (at least) two-track MIDI files. The
first track consists of the static memory section and the second of the code
section. Other tracks are ignored. 

Within a track, interpretation is ignored until one whole note (4 beats in 4/4)
is found by the Ballad VM. All Ballad tracks must have this whole note at the
beginning of a valid section, code or data. This note is used to decode the
rest of the section, by providing the appropriate key, or MIDI note number.
This key is used to demodulate the rest of the song, which is restricted to
notes on the major scale, based on this key.

Demodulation occurs by first transposing the key of song to a common key
(typically by subtracting the MIDI number of the tonic note mentioned above
from the MIDI number of the given note to decode). The only relevant note in a
chord is its root; all higher notes are ignored in this version of Ballad. Once
the song is in a uniform key, it is necessary to obtain the offset \emph{in the
scale} of the given note to decode (as opposed to its raw MIDI number). In any
given Ballad program, there are only sixteen notes of the major scale given as
chord roots or individual notes. To obtain Ballad bytecode from a Ballad song,
it is necessary to map these to bytes.

The mapping of notes to bytes is done by first obtaining the aforementioned
offset. This yields one of sixteen numbers, or 0x0 to 0xF. Taken in pairs, from
the beginning of the Ballad song to the end, then converted into bytes, Ballad
bytecode is obtained. For example, if the whole note at the start of one of the
tracks was `A-0', then the next two notes were `A-0' then `D-1', the resulting
hex byte would be 0x0a, or newline in ASCII (where 0 represents the first
octave, 1 represents the one above it). By applying this to each of the first
two tracks of the Ballad song, a sequence of bytes can be obtained.

\subsection{Static Memory} Static memory is used to provide predetermined input
to a Ballad program. In other languages, this would usually take the form of
constant character strings. It is indexed from zero by Ballad instructions and
can be loaded via the static loading commands.

\subsection{Code Memory} Code memory is a linear chain of instructions and
their arguments. There are instructions which take between one and three
arguments. Each instruction to a Ballad argument is one byte long, which makes
calls fairly straightforward.

\clearpage
\section{Ballad Instructions} This section describes the various Ballad
instructions, their calling conventions and what registers are accessed by a
given call, if any. Math registers are referenced as \emph{m}, the general
purpose, or swap registers, are referenced by \emph{s} and immediate
instructions (in unsigned single-byte representations) are indicated by
\emph{im}. If an instruction accepts no arguments, \emph{None} is used.

\subsection{Mathematical Instructions}
All arithmetic operations imply unsigned integer arithmetic and overflow where
appropriate.
\\
\begin{table}[h!]
\centering
\caption{Mathematical Instructions}
\begin{tabular}{| l | l | l | p{8cm} | }
	\hline
	Instruction & Opcode & Arguments & Semantics \\\hline
		       add & 1 & m0, m1 & Add the values stored in m0 to m1 and store the result
	in m0 \\\hline
		       sub & 2 & m0, m1 & Subtract the values stored in m1 from m0 and store the result
	in m0 \\\hline
		       mul & 3 & m0, m1 & Multiply the values stored in m0 to m1 and store the result
	in m0 \\\hline
		       div & 4 & m0, m1 & Divide m0 by m1 and store the result
	in m0 \\\hline
		       xor & 5 & m0, m1 & Perform a bitwise exclusive or on the values stored in   
	m0 to m1 and store the result in m0 \\\hline
		       and & 6 & m0, m1 & Perform a bitwise `and' on the values stored in   
	m0 to m1 and store the result in m0 \\\hline
		       or & 7 & m0, m1 & Perform a bitwise `or' on the values stored in   
	m0 to m1 and store the result in m0 \\\hline
		       or & 8 & m0, m1 & Perform a bitwise inverse (0b00000000 $\rightarrow$ 0b11111111) on the 
	values stored in   
	m0 to m1 and store the result in m0 \\\hline
	inc & 9 & m0 & Increment the value in m0 by one and save it to m0 \\\hline
	dec & 10 & m0 & Decrement the value in m0 by one and save it to m0 \\\hline
\end{tabular}
\end{table}
\clearpage
\subsection{Jump Instructions}
These instructions change the instruction pointer (IP) in some way. Offsets
range from 0 (the start of the code section) and can end at 255 (the end of the
code section). When given, an immediate (\emph{im}) value is an offset.
\begin{table}[h!]
	\centering
	\caption{Jump Instructions}
	\begin{tabular}{|l|l|l|p{8cm}|}
		\hline
		Instruction & Opcode & Arguments & Semantics \\\hline
			       jmp & 11 & im & Set the IP to the given offset \\\hline
			       jeq & 12 & m0, m1, im & If the value in m0 is equal to the value stored
					in m1, set the IP to the offset \\\hline
			       jne & 13 & m0, m1, im & If the value in m0 is not equal to the value stored
					in m1, set the IP to the offset \\\hline
			       jlt & 14 & m0, m1, im & If the value in m0 is less than the value stored
					in m1, set the IP to the offset \\\hline
			       jgt & 15 & m0, m1, im & If the value in m0 is greater than the value stored
					in m1, set the IP to the offset \\\hline
			       jlte & 16 & m0, m1, im & If the value in m0 is less than 
					or equal to the value stored
					in m1, set the IP to the offset \\\hline
			       jgte & 17 & m0, m1, im & If the value in m0 is greater than 
					or equal to the value stored
					in m1, set the IP to the offset \\\hline
				ret & 28 & m0 & Set the IP to the value stored in m0 \\\hline
	\end{tabular}
\end{table}

\clearpage
\subsection{Memory Instructions}
These instructions deal with moving data in and out of static memory and the
stack. Typically, the value to move is stored in an \emph{s} register and the
offset in the target memory location is stored in an \emph{m} register.
\begin{table}[h!]
	\centering
	\caption{Memory Instructions}
	\begin{tabular}{|l|l|l|p{8cm}|}
		\hline
		Instruction & Opcode & Arguments & Semantics \\\hline
		       push & 18 & s0 & Put the value stored in s0 at the top of the stack, then 
		increment the stack pointer \\\hline
		pop & 19 & s0 & put the value on the top of the stack into s0, then decrement 
		the stack pointer\\\hline
lstat & 23 & s0, m0 & Fetch the byte stored in static memory from the offset stored in m0 and put it into
		s0 \\\hline
		stget & 24 & s0, m0 & Fetch the value stored on the stack at the offset stored in
		m0 and put it into s0 \\\hline
		stput & 25 & m0, s0 & Put the value stored in s0 onto the stack at the offset stored
		in m0\\\hline
	\end{tabular}
\end{table}

\clearpage
\subsection{Move Instructions}
These instructions handle moving data between registers and putting immediate
data into registers. They also handle things like immediate loading from within
the code. As usual, assume unsigned bytes.
\begin{table}[h!]
	\centering
	\caption{Move Instructions}
	\begin{tabular}{|l|l|l|p{8cm}|}
		\hline
		Instruction & Opcode & Arguments & Semantics \\\hline
		      movim & 30 & m0, im & This stores the value of the immediate into the math register
		indexed by m0 \\\hline
		movis & 29 & s0, im & This stores the value of the immediate into the swap register
		indexed by s0 \\\hline
		movrm & 21 & m0, s0 & This moves the value stored in the swap register s0 into the 
		math register indexed by m0 \\\hline
		movrs & 21 & s0, m0 & This moves the value stored in the math register m0 into the 
		swap register indexed by s0 \\\hline
	\end{tabular}
\end{table}

\clearpage
\subsection{Utility Instructions}
These instructions handle `system' level instructions, reading and writing from the user.
\begin{table}[h!]
	\centering
	\caption{Move Instructions}
	\begin{tabular}{|l|l|l|p{8cm}|}
		\hline
		Instruction & Opcode & Arguments & Semantics \\\hline
		       read & 27 & m0, m1 & This command reads in bytes numbering the value stored
		in m1 onto the stack at the address (offset) given by the value stored in m0\\\hline
		       print & 26 & s0, s1 & This command prints out bytes numbering the value stored
		in s1 from the stack at the address (offset) given by the value stored in s0\\\hline
	\end{tabular}
\end{table}

